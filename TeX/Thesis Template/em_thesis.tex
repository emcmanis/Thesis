% This is the Reed College LaTeX thesis template. Most of the work 
% for the document class was done by Sam Noble (SN), as well as this
% template. Later comments etc. by Ben Salzberg (BTS). Additional
% restructuring and APA support by Jess Youngberg (JY).
% Your comments and suggestions are more than welcome; please email
% them to cusf@reed.edu
%
% See http://web.reed.edu/cis/help/latex.html for help. There are a 
% great bunch of help pages there, with notes on
% getting started, bibtex, etc. Go there and read it if you're not
% already familiar with LaTeX.
%
% Any line that starts with a percent symbol is a comment. 
% They won't show up in the document, and are useful for notes 
% to yourself and explaining commands. 
% Commenting also removes a line from the document; 
% very handy for troubleshooting problems. -BTS

% As far as I know, this follows the requirements laid out in 
% the 2002-2003 Senior Handbook. Ask a librarian to check the 
% document before binding. -SN

%%
%% Preamble
%%
% \documentclass{<something>} must begin each LaTeX document
\documentclass[12pt,twoside]{reedthesis}
% Packages are extensions to the basic LaTeX functions. Whatever you
% want to typeset, there is probably a package out there for it.
% Chemistry (chemtex), screenplays, you name it.
% Check out CTAN to see: http://www.ctan.org/
%%
\usepackage{graphicx,latexsym} 
\usepackage{amssymb,amsthm,amsmath}
\usepackage{longtable,booktabs,setspace} 
\usepackage{subfig}
%\usepackage{chemarr} %% Useful for one reaction arrow, useless if you're not a chem major
\usepackage{url}
\usepackage[numbers]{natbib}
\usepackage{pdfpages}
\usepackage{fancyvrb}
\usepackage{multirow}
\usepackage{chngpage}
% \usepackage{times} % other fonts are available like times, bookman, charter, palatino

\newcommand{\eqn}[1]{\begin{equation}#1\end{equation}}
\newcommand{\eq}[1]{\begin{align}#1\end{align}}
\newcommand{\fig}[2]{\begin{figure}\begin{center}#1\end{center}#2\end{figure}}

\def\changemargin#1#2{\list{}{\rightmargin#2\leftmargin#1}\item[]}
\let\endchangemargin=\endlist


\title{Quantum Mechanical Bound States of the Yukawa Potential} 
\author{Ellen M. McManis}
% The month and year that you submit your FINAL draft TO THE LIBRARY (May or December)
\date{May 2012}
\division{Mathematics and Natural Sciences}
\advisor{Nelia Mann}
%If you have two advisors for some reason, you can use the following
%\altadvisor{Your Other Advisor}
%%% Remember to use the correct department!
\department{Physics}
% if you're writing a thesis in an interdisciplinary major,
% uncomment the line below and change the text as appropriate.
% check the Senior Handbook if unsure.
%\thedivisionof{The Established Interdisciplinary Committee for}
% if you want the approval page to say "Approved for the Committee",
% uncomment the next line
%\approvedforthe{Committee}

\setlength{\parskip}{0pt}
%%
%% End Preamble
%%
%% The fun begins:
\begin{document}

  \maketitle
  \frontmatter % this stuff will be roman-numbered
  \pagestyle{empty} % this removes page numbers from the frontmatter

% Acknowledgements (Acceptable American spelling) are optional
% So are Acknowledgments (proper English spelling)
    \chapter*{Acknowledgements}
In the completion of this thesis, a profound debt is owed:

To my adviser, Nelia Mann, for all the help, support, and knowledge. To Mary James, for the help learning how to learn.

To my friends. To my classmates and fellow Reedies, who shaped me into the person who wrote this. To the upperclassmen (Reuven, Rosie, Robin, and Mary) who took the time, my freshman year, to tell me how neat physics was. To Joel Franklin, for sharing neat problems and his excitement for the subject. 

To everyone on IRC who's been listening to my thesis ramblings these past months. And finally, to those people who have loved and supported me through this year.

You people are amazing.
% The preface is optional
% To remove it, comment it out or delete it.
%    \chapter*{Preface}
%	This is an example of a thesis setup to use the reed thesis document class.

    \tableofcontents
% if you want a list of tables, optional
    \listoftables
% if you want a list of figures, also optional
    \listoffigures

% The abstract is not required if you're writing a creative thesis (but aren't they all?)
% If your abstract is longer than a page, there may be a formatting issue.
    \chapter*{Abstract}
We obtain a model for the relationship between the number of bound states of the Yukawa potential and $\lambda = \ell/\alpha$, the ratio of the exponential cutoff length scale to the ``Bohr radius'' for the problem. We use a semi-classical approximation to make a qualitative prediction for the relationship. We then use the finite difference method to obtain numerical values for the critical values $\lambda_n$, which are the values of $\lambda$ at which the $n$th new bound state appears. These can be fit to a quadratic function, giving us $\lambda_n = A(l) + B(l)n + C(l)n^2$, where $A$, $B$, and $C$ are constants that depend on the angular momentum quantum number $l$. 
%	\chapter*{Dedication}
%	You can have a dedication here if you wish.

  \mainmatter % here the regular arabic numbering starts
  \pagestyle{fancyplain} % turns page numbering back on

%The \introduction command is provided as a convenience.
%if you want special chapter formatting, you'll probably want to avoid using it altogether

    \chapter*{Introduction}
         \addcontentsline{toc}{chapter}{Introduction}
	\chaptermark{Introduction}
	\markboth{Introduction}{Introduction}
	% The three lines above are to make sure that the headers are right, that the intro gets included in the table of contents, and that it doesn't get numbered 1 so that chapter one is 1.

% Double spacing: if you want to double space, or one and a half 
% space, uncomment one of the following lines. You can go back to 
% single spacing with the \singlespacing command.
% \onehalfspacing
% \doublespacing
The Yukawa potential describes a force mediated by a massive particle. It is %argh fuck this part
\eqn{
V(r) = -\frac{C}{r}e^{-r/\ell}\mbox{,}
}
where $C$ and $\ell$ are constants. $C$ sets the strength of the force; $\ell$ acts as a length scale. The exponential term provides an effective cutoff for the associated force: once $r$ gets much larger than $\ell$, the exponential term drops off quite rapidly. We are interested in how the bound states of a particle subject to this potential change with these constants. 

The length scale of the potential, which limits the range at which the force acts, is related to the mass of the mediating particle $m$ by the equation\eqn{
\ell = \frac{\hbar}{m c}\mbox{.}
\label{eq:ell}
}
Both the mass $m$ and the strength of the force $C$ depend on the application of the potential. Most commonly, we use this potential to describe the force between protons and neutrons in an atomic nucleus, where $m$ is the mass of the pion.
 
In general, to find the bound states of a particle under the influence of some $V(r)$, we solve the time-independent Schr\"odinger wave equation,
\eqn{
-\frac{\hbar^2}{2\mu}\nabla^2\psi(r,\phi,\theta) + V(r)\psi (r,\phi,\theta) = E \psi(r,\phi,\theta)\mbox{,}
\label{eq:TIDSWE-general}
} and look for solutions which go to zero at infinity.
%In this thesis, we are working with a system of two particles attracted to each other by the Yukawa potential. To simplify the system, it makes the most sense to express this in spherical coordinates as a reduced mass orbiting a center of mass. This is the form used by the wave equation above. 
With the Yukawa potential, this equation cannot be solved analytically. However, in the limit where $r \ll \ell$, the exponential term disappears, leaving
\eqn{
V(r) \approx -\frac{C}{r}\mbox{.}
\label{eq:vapprox}
}
This is the same form as the Coulomb potential, which we use to find the bound states of an electron and a proton in hydrogen (for $C = \frac{e^2}{4\pi\epsilon_0}$). \eqref{eq:vapprox} suggests that we will be able to gain some insight into the Yukawa potential by looking at the solutions to the wave equation for the Coulomb potential. 

These solutions can be written as \cite{griffiths_introduction_2005}
\eqn{
\psi_{l, m_l, n} (r, \theta, \phi) = R_{n,l}(r) Y_{l,m_l}(\theta,\phi)\mbox{.}
}
Here, $l$, $m_l$, and $n$ are quantum numbers. The $Y_{l, m_l}$ are the spherical harmonics. For simplicity, we begin by focusing on states with zero angular momentum, for which the solutions are spherically symmetric.
For convenience, we introduce a length scale, 
\eqn{
\alpha = \frac{\hbar^2}{C \mu}\mbox{,}
\label{eq:bohrradius}
}
which is a constant with units of length. 
In hydrogen this is the Bohr radius, and reduces to the familiar $a_o = \frac{4\pi \epsilon_0 \hbar^2}{\mu e^2} = 5.3 \times 10^{-11}$ m. 

The simplest nucleus we might use the Yukawa potential for is that of deuterium. In deuterium, $C =  1.10286 \times 10^{-26}$ J m\footnote{This value was obtained by using the numerical methods described in Chapter 2 to work backward from the known value of the length scale and the mass defect of deuterium, which will be equal to the energy of the ground state of the system.} and $\mu = m_p/2$, giving an $\alpha$ of $1.20 \times 10^{-15}$ m. This is interestingly on the same order as $\ell = 1.46 \times 10^{-15}$, the cutoff length scale, which we obtain from \eqref{eq:ell} using $m = m_{\pi} = 140$.

We can now take $Y_{0,0} = 1/\sqrt{4 \pi}$, leaving us with
\eqn{
\psi_{n, l =0}(r) = \frac{1}{\sqrt{4\pi}}R_{n , l= 0}(r) = \frac{1}{\sqrt{4\pi}} e^{-r /\alpha n}L_{n-1} \left(2\frac{r}{ \alpha n}\right)
\label{eq:SWE-coulomb}
}
where $L_{n-1}$ is a Laguerre polynomial. The factor of $e^{-r /\alpha n}$ ensures that the wave function goes to 0 as $r$ goes to infinity, with $n$ controlling how fast that happens. The Laguerre polynomial determines how many nodes the wave function has. Because $L_{n-1}$ has $n-1$ roots, so the wave function will have $n-1$ nodes (figure \ref{fig:wavefunctions}).
\begin{figure}[h]
\centering
\subfloat[$n = 1$]{\includegraphics[width=0.3\textwidth]{Figures/ps0}}
\subfloat[$n = 2$]{\includegraphics[width=0.3\textwidth]{Figures/ps1}}
\subfloat[$n = 3$]{\includegraphics[width=0.3\textwidth]{Figures/ps2}}
\caption[The first three bound states of $\psi(\rho)$ for the Coulomb potential]{The first three bound states of $\psi(\rho)$ for the Coulomb potential, where $\rho = r/\alpha$. These show the nodes given by the Laguerre polynomials as well as the exponential dieoff of the wave function.}
\label{fig:wavefunctions}
\end{figure}

The energy for the particle described by this wave function is just a function of $n$ (the principal quantum number), and is equal to
\eqn{
E_n = -\frac{C}{2\alpha n^2}\mbox{.}
\label{eq:henergy}
}
There are an infinite number of bound states labelled by $n$, and as $n$ gets large, the energies given by \eqref{eq:henergy} get closer and closer to zero but never quite reach it (the states remain bound). As this happens, the wave functions can be found further and further from the origin. This effect can be seen in figure \ref{fig:boundstates}, in which the first four bound states are plotted.

\begin{figure}[h]
\centering
\includegraphics{Figures/densityplots}
\caption[Density plots of the first four spherically symmetric Coulomb wave functions]{The first four spherically symmetric wave functions. This plot shows the probability density, $\psi(\rho)^2$, in the x-y plane. Again, $\rho = r/\alpha$ is the non-dimensionalized radial coordinate.}
\label{fig:boundstates}
\end{figure}

Now let us consider how this changes for the Yukawa potential, which itself has an exponential die-off with length scale $\ell$. Because of \eqref{eq:vapprox}, we expect the Yukawa states to resemble the Coulomb states for bound states with $r \ll \ell$. For $r \gg \ell$, the potential is approximately $0$. We do not expect to find any bound states where the wave function is significantly non-zero in this region, because there is no force to bind the particle. We therefore expect that the number of bound states of this potential will be be finite.

When working with the Yukawa problem, we will continue using the length scale $alpha$ from \eqref{eq:bohrradius}. What we are interested in is not $\ell$ itself, but the ratio $\ell / \alpha$ which we call $\lambda$.
If $\lambda$ is small, a given state is more likely to fall outside of the range of the potential, meaning that only a small number of states (those whose wave functions do not extend very far out) may be bound. For a large $\lambda$, the reverse is true, and we expect a large number of bound states.

To find the relationship between the number of bound states and $\lambda$, we will use the finite difference method to solve the Schr\"odinger equation in chapter 2. Because we are explicitly working around the area where $r \approx \ell$, an analytic approximation by perturbing the Coulomb potential \cite{PhysRevLett.66.1310} will not give us much insight into what we expect our numerical results to be. Instead, in chapter 1 we use the Bohr method to obtain an idea of what we expect.

%\begin{figure}[h]
%\begin{center}
%\includegraphics[scale=0.7]{Figures/hydrogenspectrum}
%\end{center}
%\caption{The energies of the first 10 bound states of the electron in the hydrogen atom}
%\label{fig:hspec}
%\end{figure}
\clearpage %% starts a new page and stops trying to place floats such as tables and figures

\chapter{Semi-Classical Approximation}
To better understand the relationship between the number of bound states and $\ell$, we begin with a semi-clasical approximation. This will give us an idea what to expect from the numerical analysis in chapter 2.
\section{The Bohr Model}
For this approximation we can again exploit the similarities between the Yukawa potential and the Coulomb potential and use the Bohr model. 
%The Bohr model is most commonly applied to the electron orbiting the nucleus in hydrogen. 
The postulates of the model include the following:
\begin{enumerate}
\item Particles move in circular orbits around the center under the influence of some potential $V(r)$. They obey all the laws of classical mechanics.
\item Angular momentum is quantized; $L = n\hbar$ where $n$ is a positive integer.
\end{enumerate}

To use these postulates, first we look at the classical energy of a particle moving under the influence of a potential:
\eqn{
E = \frac{1}{2} \mu \dot{r}^2 + \frac{L^2}{2 \mu r^2} + V(r) = \frac{1}{2}\mu \dot{r}^2+\frac{1}{2}\frac{(n \hbar)^2}{\mu r^2}+V(r)\mbox{.}
\label{eq:classical-energy}
}

We can roll the second half of the equation up into an effective potential, producing something of the form $E = KE + PE$. The effective potential $V_{\mathrm{eff}}$ is then
\eqn{
V_{\mathrm{eff}}(r)=\frac{(n \hbar)^2}{2 \mu}\frac{1}{r^2}+V(r)\mbox{.}
\label{eq:veff}
}
Referring back to the first postulate, we know that our semi-classical particle is moving in a circular orbit. Classically, a circular orbit occurs when the effective potential is minimized, that is, when
\eqn{
\frac{dV_{\mathrm{eff}}}{d r}\Big |_{r = r_0}= 0\mbox{.}
\label{eq:vrnaught}
}
Given some $n$, the radius of the Bohr orbit will be given by \eqref{eq:vrnaught}. The energy of this orbit can be obtained by evaluating \eqref{eq:classical-energy}
 at $r_0$. Because the orbit is circular, $\dot{r} = 0$, so the energy will be 
\eqn{
E_{n} = \frac{1}{2}\frac{(n\hbar)^2}{\mu r_0^2} - V(r_0)\mbox{.}
}

\section{Application to Coulomb Potential}
Let's remind ourselves how this works for the Coulomb potential. To make this easier to work with, we define a nondimensionalized radial coordinate
\begin{equation}
\rho \equiv \frac{r}{\alpha}
\label{eq:rho}
\end{equation}
using the length scale from \eqref{eq:bohrradius}, the ``Bohr radius'' for this problem.

We also define a time scale:
\begin{equation}
\beta \equiv \frac{C^2m}{\hbar^3}\mbox{.}
\label{eq:beta}
\end{equation}
We can substitute these into \eqref{eq:classical-energy} and \eqref{eq:veff} in order to obtain 
\eqn{
\frac{2 \alpha}{C}E = \tilde{E}= \rho '^2 + \frac{n^2}{\rho^2}-\frac{2}{\rho}
\label{eq:energy-nondim}
}
\eqn{
\tilde{V}_{\mathrm{eff}} = \frac{n^2}{\rho^2}-\frac{2}{\rho}
\label{eq:veff-nondim}
}

We can plot the effective potential as a function of $\rho$ (figure \ref{fig:hveff}). The circular orbits will be at the minimum of this effective potential, where the radius of the particle's orbit is constant. The value of the energy is then $\tilde{V}_{\mathrm{eff,min}}$. Note that a different value of $n$ gives a different effective potential with a different minimum. 

\begin{figure}[h]
\center{\includegraphics[width=\textwidth]{Figures/hyd_Veff}}
\caption[The effective Coulomb potential]{The effective Coulomb potential. Particles can exist with energies greater than the potential energy, and their trajectories will cover all possible radii at that energy. The black dots on the $\rho$ axis indicate the minima of the curves, where circular orbits exist. The dots on the $V_{\mathrm{eff}}$ axis indicate the values of the energy at those minima.}
\label{fig:hveff}
\end{figure}

To find the location of this minimum, we differentiate \eqref{eq:veff-nondim} and set the result to $0$, which gives
\eqn{
\rho_0 = n^2
\label{eq:rho-n}
}
Plugging this into \eqref{eq:energy-nondim}, we find that the energy of the particle in this orbit is
\eqn{
\tilde{E} = -\frac{1}{n^2}
}
For the Coulomb potential, this is perfect -- the semi-classical Bohr model gives exactly the same energy spectrum as we would obtain by solving the Schr\"odinger equation, despite the inappropriate use of classical mechanics. In other quantum systems, a Bohr approach generally gives a qualitatively reasonable result that's off by a bit quantitatively. For example, when the three-dimensional simple harmonic oscillator is studied, the Bohr method obtains the right form of the spectrum, but it differs from the actual spectrum by an offset of the ground state energy. For the logarithmic potential, the Bohr spectrum (with appropriate nondimensionalization) is $\tilde{E}_n = \frac{1 + \ln(2)}{2}+\ln(n)$, while the numerically determined quantum spectrum is in the form $\tilde{E}_n = A + \ln(n+B)$ for some constants $A$ and $B$ \cite{Garon}.

\section{Approximating the Yukawa Potential with a Cutoff}

Returning to the the Yukawa potential, when $r << l$ we expect the $\frac{1}{r}$ behavior to dominate, and the potential to behave similarly to the Coulomb potential. When $r >> l$, we expect the exponential term to dominate, and the potential to go to zero. As a na\"ive approximation of this behavior, we can consider cutting off the Coulomb potential at $r = l$. In keeping with the nondimensionalization earlier, we define $\lambda \equiv \frac{l}{\alpha}$. Our effective potential is now
\eqn{
\tilde{V}_{\mathrm{eff}}(\rho) = \left\{
\begin{array}{lr}
 \frac{n^2}{2}\frac{1}{\rho^2}-\frac{1}{\rho} & : \rho \leq \lambda \\
0 & : \rho > \lambda
\end{array}
\right.
\label{eq:naive}
}

\fig{
\includegraphics{Figures/cutoff}}
{
\caption[Yukawa orbits with cutoff]{A polar plot of the first three Bohr orbits along with a cutoff at $\lambda = 2.5$. Because the $n=3$ orbit lies outside the cutoff, it would not be allowed under \eqref{eq:naive}.}
\label{fig:cutoff}
}

This cutoff means that the relationship between $\rho$ and $n$ holds while $\rho < \lambda$, but breaks down after, limiting the number of bound states (figure \ref{fig:cutoff}). Any bound state allowed under \eqref{eq:rho-n} for which $\rho$ exceeds $\lambda$ will not exist. If we imagine increasing $\lambda$ gradually, a new bound state will be added whenever we pass through a point $\lambda = n^2$ (for some integer $n$). We call these points the critical values of lambda, notated $\lambda_{n}$, where $n$ is a positive integer that corresponds to the number of bound states just above that critical value. Based on this treatment, we expect $\lambda_{n} = n^2$.

\section{Application to Yukawa Potential}
To perform a more careful analysis of the potential, we look at the graph of its effective potential (figure \ref{fig:yveff}):
\eqn{
\tilde{V}_{\mathrm{eff}} = \frac{n^2}{\rho^2} - \frac{2}{\rho}e^{-\rho/\lambda}
\label{eq:yveff}
}
 
As before, we can see that bound states exist for some $n$ and $\rho$ where $V_{\mathrm{eff}}$ is less than 0, and these bound states have circular orbits at the minimum of $V_{\mathrm{eff}}$. However, there also appear to be bound states \emph{above} $E = 0$. Clasically, these are perfectly valid elliptical orbits. Quantum mechanically, a particle in one of these states can tunnel through and become a scattering state. This gives us two possible definitions for a critical value of $\lambda$.  The first is where the minimum of $V_{\mathrm{eff}}$ for some $n$ moves below zero -- that is, that value of lambda allows an additional stable bound state ($E < 0$). The other occurs where $V_{\mathrm{eff}}$ for some $n$ first starts to have a minimum at all, giving it a bound state with $E>0$. These two critical points are shown in figure \ref{fig:critpoints}

\begin{figure}[h!]
\center{\includegraphics[width=\textwidth]{Figures/yukawa_Veff}}
\caption[The effective Yukawa potential]{The effective Yukawa potential. \eqref{eq:yveff} was plotted for $n = 1$ through $6$ with $\lambda = 32$. As with Coulomb, the curves get increasingly shallow as $n$ increases. The curves for $n = 1$ through $4$ all have a circular orbit with $E < 0$, that is, they have a stable bound state. The curve for $n = 5$ has a very slight minimum, but it does not go below $0$, meaning this corresponds to an unstable bound state. The $n = 6$ curve does not have a minimum at all. For all curves with a minimum, the approximate location of this minimum has been plotted on the $\rho$ axis and the value of the energy at this minimum on the $V_{\mathrm{eff}}$ axis.}
\label{fig:yveff}
\end{figure}
\begin{figure}[h]
\center{\includegraphics{Figures/criticalpoints}}
\caption[The two critical values of $\lambda$]{Two curves showing $V_{\mathrm{eff}}$ at the two critical values of $\lambda$ -- one where a new stable bound state appears, and one where a new bound state with $E > 0$ appears.}
\label{fig:critpoints}
\end{figure}

\subsection{Stable bound states}
The first of these critical values occurs when the minimum of $V_{\mathrm{eff}}$ is 0. We again start by taking the derivative of \eqref{eq:yveff} and setting it to 0, obtaining
\begin{equation}
\frac{dV}{d\rho}\Big |_{\rho = \rho_0} =0 = -\frac{2n^2}{\rho_0^3} + \frac{2}{\rho_0^2}e^{-\rho_0/\lambda} + \frac{2}{\rho_0 \lambda} e^{-\rho_0/\lambda}
\label{vmin}
\end{equation}
where $\rho_0$ is the value of $\rho$ when $V_{\mathrm{eff}}$ is minimized. This equation holds for any $\lambda$ has no general solution, but we can find one for the specific case where $\lambda$ is some critical value $\lambda_n$ and $V_{\mathrm{eff}}(\rho_0) = 0$ by solving the equation
\begin{equation}
V_{\mathrm{eff}}(\rho_0)=0 = n^2 - 2\rho_0 e^{-\rho_0/\lambda_n} 
\end{equation}
for $n^2$
and plugging the result into \eqref{vmin}, obtaining
\begin{equation}
\rho_0 = \lambda_n \mbox{.}
\end{equation}
Plugging this back into the equation for $V_{\mathrm{eff}}(\rho_0)$, we find that
\begin{equation}
\lambda_n=\frac{e}{2}\, n^2\approx 1.36\, n^2
\label{eq:stable}
\end{equation}
which is pretty close to the behavior $\lambda_{n} = n^2$ from the cutoff Coulomb potential above. 

\subsection{Unstable bound states}

Having determined a formula for the $\lambda_n$s at which new bound states with $E < 0$ appear, we turn to the appearance of classically bound states with $E > 0$. A bound state exists when $V_{\mathrm{eff}}$ has a minimum. We found the $\rho_0$ where this minimum occurs in \eqref {vmin}, which we can rewrite as
\begin{equation}
n^2 = \left( \rho_0 + \frac{\rho_0^2}{\lambda} \right) e^{-\rho_0/\lambda}
\label{eq:unstablemins}
\end{equation}
The right side of this equation will have some maximum, which will be the last point for which the equation can be solved -- after that, the value of $n^2$ will always be greater than the right side. This is shown in figure \ref{fig:unstableveff}. For a given $n$, this gives $\lambda_n$.
\begin{figure}[h]
\centering
\includegraphics{Figures/unstableveff}
\caption[$n^2$ as a function of $\rho_0$]{This is a plot of \eqref{eq:unstablemins} with $\lambda = 2$, showing its maximum value. At this value of $\lambda$, the $n = 1$ bound state exists, because for $n = 1$ there is some $\rho_0$ that satisfies \eqref{eq:unstablemins}. This is not the case for the $n = 2$ bound state, so the $n = 2$ bound state does not exist. Increasing $\lambda$ scales the curve, increasing the number of bound states.}
\label{fig:unstableveff}
\end{figure}
To find this maximum, we differentiate with respect to $\rho_0$ and set the result 0, obtaining
\begin{equation}
0 = (1+2 \frac{\rho_0}{\lambda_n}) - \frac{1}{\lambda_n}(\rho_0 + \frac{\rho_0^2}{\lambda_n})\mbox{.} 
\end{equation}
This is a quadratic in $\rho_0$ and can be solved easily.
As $\rho_0$ is a real physical quantity that can't be negative, we can discard one solution, obtaining
\begin{equation}
\rho_0 = \frac{\lambda_n}{2}(1+\sqrt{5})\mbox{.}
\end{equation}
We can plug this in to \eqref{vmin} to find the relationship between $\lambda$ and $n$:
\begin{equation}
n^2 = \lambda_{n}(2 + \sqrt{5})e^{-(1+ \sqrt{5})/2} \Rightarrow \lambda_n =  1.19\, n^2
\label{eq:unstable}
\end{equation}
\section{Summary}
Using the Bohr model with various assumptions to analyze the Yukawa potential yields three different expressions for the critical values $\lambda_n$: \eqref{eq:naive}, \eqref{eq:stable}, and \eqref{eq:unstable}. Each of these has the form 
\eqn{
\lambda_{n} = a n^2
}
where $a$ is some constant of order one. This suggests that when performing our numerics, we will find critical values of lambda that are proportional to the square of the number of bound states. 

When applied to deuterium, we have $\lambda = 1.21245$. This suggests deuterium has exactly one bound state. However, it also implies that this bound state is unstable with $E > 0$. Given that there is a mass defect observed in deuterium, this can't quite be right. As mentioned earlier, the Bohr method does a good job of capturing qualitative behavior, but is often quantitatively inaccurate. This discrepancy between model and reality is evidently an example of the method's limitations.

\clearpage %% starts a new page and stops trying to place floats such as tables and figures

\chapter{Numerical Methods and Results}

In the previous chapter, we used semi-classical methods to analytically obtain an estimate for the number of bound states allowed by the Yukawa potential for a given $\lambda$.  We now turn our attention to the Schr\"odinger Wave Equation in order to numerically determine the critical values of $\lambda$ at which a new bound state appears. We will first focus on states with no angular momentum, which have spherically symmetric bound states. We will then generalize to states with non-zero angular momentum.

\section{The Hamiltonian}

Schr\"odinger's equation for states $\psi$ with definite energy $E$ can be written as
\eqn{
H \psi = E\psi
\label{eq:hamiltonian}
}
where $H$ is the Hamiltonian operator, $H =-\frac{\hbar^2}{2m}\nabla^2 +V$.  We will begin by focusing on spherically symmetric bound states, so we will only be concerned with the $r$ component of the $\nabla$ operator. \eqref{eq:hamiltonian} then becomes
\eqn{
H = -\frac{\hbar^2}{2m} \left[\frac{1}{r^2}\frac{d}{dr}\left(r^2 \frac{d}{dr}\right)\right] - \frac{C}{r}e^{-r/\ell}\mbox{,}
}
plugging in the Yukawa potential for $V$.

Using the same nondimensionalizations as in last chapter, we rewrite this as 
\eqn{
H=\frac{C}{2\alpha}\tilde{H} = \frac{C}{2\alpha} \left[- \frac{1}{\rho^2}\frac{d}{d\rho}\left(\rho^2\frac{d}{d\rho}\right) - \frac{2}{\rho}e^{-\rho/\lambda}\right]\mbox{.}
\label{eq:hnondim}
}
The constant out front is the same one we used when we nondimensionalized the energy of the system, which allows us to restate \eqref{eq:hamiltonian} as
\eqn{
\tilde{H}\psi = \tilde{E}\psi\mbox{.}
}
To simplify, we can substitute 
\eqn{
u(\rho)=\rho \phi(\rho)}
into the expression in \eqref{eq:hnondim}\cite{griffiths_introduction_2005}. This gives us a much neater Hamiltonian, and our differential equation is now
\eqn{
-\frac{d^2 u }{d \rho^2} - \frac{2}{\rho}e^{-\rho/\lambda}u = \tilde{E} u\mbox{.}
\label{eq:hamiltonianfinal}
}

\section{Numerics: The Finite Difference Method}
The finite difference method transforms a differential equation into a matrix problem\cite{numerical_recipes}. It does this by approximating the first and second derivatives of a function on a discrete grid of points $\rho_n$ which are a distance $\Delta \rho$ apart. The function $u(\rho)$ then becomes a function $u(\rho_n)$. The grid runs from $\rho_1$ to some $\rho_N$. Using the concept of the derivative as the slope of the tangent line at some point, we can write
\eqn{
u'(\rho_n) = \frac{u(\rho_n + \Delta \rho) - u(\rho_n - \Delta \rho)}{2 \Delta \rho} - O(\Delta{\rho}^2)
\label{fdifference}
}
\eqn{
u''(\rho_n) = \frac{u(\rho_n + \Delta \rho) - 2 u(\rho_n) + u(\rho_n - \Delta \rho)}{\Delta \rho^2}-O(\Delta{\rho}^2) 
\label{f2difference}
}
Applying \eqref{f2difference} to our differential equation in \eqref{eq:hamiltonianfinal}, we obtain
\eqn{
-u_{n-1}\frac{1}{\Delta \rho^2} - u_n\left(\frac{2} {\Delta \rho^2} -  \frac{2}{\rho_n}e^{-\rho_n/\lambda} \right) - u_{n-1}\frac{1}{\Delta \rho^2}  = E u_n\mbox{.}
\label{eq:hamiltonian-disc}
}
For the interior points, there are no concerns, but at the beginning we require a $u_0$ and at the end we need a $u_{N+1}$ in order to satisfy \eqref{eq:hamiltonian-disc}. For our problem, we set $u_0 = u(0)$  and $u_{N+1} = u(\infty)$. Our substitution $u(\rho) = \rho \psi(\rho)$ ensures that $u(0) = 0$, and our assumption that we are working with a bound state implies that $u(\infty) = 0$. Because we cannot practically calculate anything infinitely far out, we define some $\rho_{\infty}$ and use $u(\rho_{\infty})$ as an approximation for $u(\infty)$. We therefore require that $u_0 = u_{N+1} = 0$.

We can now construct a Hamiltonian matrix using \eqref{eq:hamiltonian-disc}. The matrix will have entries on the off diagonal and diagonal and nowhere else:
\begin{equation*}H = \left(
\begin{array}{cccccc}
\frac{2} {\Delta \rho^2} -  \frac{2}{\rho_1}e^{-\rho_1/\lambda} & -\frac{1}{\Delta \rho^2} &  0 & ... & 0 & 0 \\
-\frac{1}{\Delta \rho^2} & \frac{2} {\Delta \rho^2} -  \frac{2}{\rho_2}e^{-\rho_2/\lambda} &  -\frac{1}{\Delta \rho^2}  & ... & 0 & 0 \\
0 &  -\frac{1}{\Delta \rho^2} & \frac{2} {\Delta \rho^2} -  \frac{2}{\rho_3}e^{-\rho_3/\lambda} &  ...  & 0 & 0 \\
 .&  . &.  &... & .& .  \\
 .& . & . & ...& .&  . \\
0 & 0 &  0 & ...&  -\frac{1}{\Delta \rho^2} & \frac{2} {\Delta \rho^2} -  \frac{2}{\rho_N}e^{-\rho_N/\lambda}   \\
\end{array}
\right)
\end{equation*}
The eigenvectors of this matrix will be the wave functions $u(\rho_n)$, and the eigenvalues will be the energies of those wave functions. For the calculation of these eigenvalues and eigenvectors, we rely on Mathematica's built-in Eigenvalues[ ] command. Mathematica uses the sparseness of the array to optimize the calculation, speeding it up immensely. 

Using this, we can now see what our wave functions actually look like. For bound states where $n^2 << \lambda$, we find that the bound states look very similar to hydrogen (figure \ref{fig:hyd-yukawa}).
\begin{figure}[h]
\centering
\includegraphics{Figures/n2hydyukawa}
\caption[The $n = 2$ bound states for the Yukawa potential and Coulomb potential]{The $n = 2$ bound states for the Yukawa potential and Coulomb potential. The Yukawa potential was calculated using my finite difference Hamiltonian with $\lambda = 20$ and  $\rho_{\infty} = 20*20 = 400$. The Coulomb wave function is the exact $n=2$ wave function, scaled by a constant to match the Yukawa function. The plot only extends out to $\rho = 30$ in order to see the earlier behavior; the function is remains 0 further out.}
\label{fig:hyd-yukawa}
\end{figure}
As $\lambda$ increases and more bound states are possible, the wave functions get more interesting, as in figure \ref{fig:largebound}.
\begin{figure}[h!]
\centering
\includegraphics{Figures/largebound}
\caption[The $n=14$ bound state at $\lambda = 200$]{The $n=14$ bound state at $\lambda = 200$, obtained with $\Delta \rho = 0.1$ and $\rho_{\infty} = 4000$. The $n=14$ Coulomb wave function is shown as well (scaled to match). At this value of $\lambda$, the $n=14$ state only recently became bound. This plot shows the whole range where anything interesting is going on -- again, the function has gone to 0 and will stay there.}
\label{fig:largebound}
\end{figure}
\section{Finding Critical Values of Lambda}
Let us define a critical value of $\lambda$ as the value of $\lambda$ where the Hamiltonian matrix gains a new negative eigenvalue. $\lambda_n$ will be the nth critical value, so that for $\lambda \geq \lambda_n$ we have at least $n$ bound states with $E < 0$. The subscript $n$ denotes the value of the principal quantum number, $n$, at this new eigenstate. This is similar to our definition in section 1.4.1, updated for the quantum mechanical case.
In order to find the critical values, we calculate the number of negative eigenvalues at each value of $\lambda$ and look for a place where that number increases using a variation on rootfinding. For a further discussion of the exact methods used, see Appendix A.

We are looking for wave functions $u(\rho)$ that behave like the ones in figure \ref{fig:scatt-to-bound}, where we plot the same state with varying $\lambda$ as we pass through a critical value. In \subref{scatt12-5} the state is unambiguously scattering. While it does go to 0 at $\rho_{\infty}$, that's meaningless -- the way we set up our numerics requires it.
The function primarily looks like that of a free particle in a box of size $\rho_{\infty}$. 
 In \subref{scatt12-9} it starts to level out, and in \subref{bound13-1} the state has become bound -- the function goes to zero before $\rho_{\infty}$ and stays there. Finally, in \subref{bound14} the state is clearly bound, going to 0 before $\rho_{\infty}$. 

\begin{figure}
\centering \subfloat[][$\lambda = 12.5$]{\label{scatt12-5}
\includegraphics[width=0.5\textwidth]{Figures/12-5scatt}}
%\hspace{8pt}% 
\subfloat[][$\lambda = 12.9$]{\label{scatt12-9}
\includegraphics[width=0.5\textwidth]{Figures/12-9scatt}} \\ 
\subfloat[][$\lambda = 13.1$]{\label{bound13-1}
\includegraphics[width=0.5\textwidth]{Figures/13-1bound}}% \hspace{8pt}%
\subfloat[][$\lambda = 14.0$]{\label{bound14}%
\includegraphics[width=0.5\textwidth]{Figures/14-bound}}
\caption[The progression of an eigenstate from scattering to bound]
{This figure shows the progression of the $n = 4$ wavefunction $u(\rho)$ from scattering to bound. All wavefunctions were calculated with $\rho_{\infty} = 40 \lambda$ and $\Delta \rho = 0.1$.} 
\label{fig:scatt-to-bound}
\end{figure}

\section{Estimating errors}
In doing our numerical analyses, we would like to have a low $\Delta \rho$ -- after all, we expect the wave function to fluctuate on lengths scales of order 1 with our non-dimensionalization. At the same time, we require $\rho_{\infty}$ to be large, because what separates bound states from scattering states is their behavior at infinity. This puts us in the frustrating numerical position of not really having anything we can skimp on to save computation time. The values of $\Delta \rho$ and $\rho_\infty$ are then are our two primary sources of error. To determine the errors from each, we look at how much our calculated value for some $\lambda_{n}$ varies with changes in $\Delta \rho$ and $\rho_{\infty}$.  

Consider a particular critical value $\lambda_{4}$. We calculated the value of $\lambda_{4}$ at four different points: $\rho_{\infty} = 40\lambda$ and $\Delta \rho = 0.1$, $\rho_{\infty} = 40\lambda$ and $\Delta \rho = 0.05$, $\rho_{\infty} = 80\lambda$ and $\Delta \rho = 0.1$, and $\rho_{\infty} = 80\lambda$ and $\Delta \rho = 0.05$. The results of this are in table \ref{tab:errorchanges}. 
\begin{table}
\centering
\begin{tabular}{r|c|c|c|}
\multicolumn{1}{r}{}
 &  \multicolumn{1}{c}{$\rho_{\infty} = 40*\lambda$}
 & \multicolumn{1}{c}{$\rho_{\infty} = 80*\lambda$}
 & \multicolumn{1}{l}{} \\
\cline{2-4}
$\Delta \rho = 0.1$ & 12.8255 & 12.7543 & $\Delta \lambda_n = 0.0712$\\
\cline{2-4}
$\Delta \rho = 0.05$ & 12.8233 & 12.7521& $\Delta \lambda_n = 0.0712$ \\
\cline{2-4}
\multicolumn{1}{r}{}
 &  \multicolumn{1}{|c|}{$\Delta \lambda_n = 0.0022$}
 & \multicolumn{1}{c|}{$\Delta \lambda_n = 0.0022$}
 & \multicolumn{1}{l}{} \\
 \cline{2-3}
\end{tabular}
\caption{Errors in $\Delta \rho$ and $\rho_{\infty}$ for $\lambda_4$.}
\label{tab:errorchanges}
\end{table}
As is seen in the table, changing $\Delta \rho$ creates the same change in $\lambda_4$ regardless of the value of $\rho_{\infty}$. The reverse is also true -- changes in $\rho_{\infty}$ produce the same change in $\lambda_4$, regardless of the value of $\Delta \rho$. We therefore concluded that these errors are independent of each other. In addition, changes in $\rho_{\infty}$ produce a much bigger change in our calculated $\lambda_{n}$ than an equivalent change in $\Delta \rho$, at least in the ranges we are considering here. However, when sufficient errors in $\rho_{\infty}$ were eliminated, errors in $\Delta \rho$ became noticeable.
 \begin{figure}[h]
\centering
\includegraphics{Figures/asymtote}
\caption[Asymtotic behavior of $\lambda_{4} (\rho_{\infty})$]{The values of $\lambda_{4}$ as $\rho_{\infty}$ is repeatedly doubled, showing their asymptotic approach to some value. $\Delta \rho$ was held constant at 0.1. The $\rho_{\infty}$ values are given in terms of $\rho_{\infty}/\lambda$ in order to make the graph more clear.}
\label{fig:asymtote}
\end{figure} 

When we doubled $\rho_{\infty}$ a few more times, we found that the calculated values of $\lambda_{n}$ appeared to asymptotically decrease, approaching some true value, as shown in figure \ref{fig:asymtote}. The size of the change in $\lambda_4$ is less than half the previous change for each doubling, indicating a rapid convergence. Because each point is closer to the true value than it is to the previous point, we can estimate the error by taking the distance between the two points. Thus, for every value of $\lambda_n$, we compute it first at some value of $\rho_{\infty}$ and $\Delta \rho$, then double $\rho_{\infty}$ and halve $\Delta \rho$. We use the value computed with the larger $\rho_{\infty}$ and smaller $\Delta \rho$ as our data point, and the difference between the two as an estimate of our error.  It is worth noting that our quoted $\lambda_n$ values will all likely be too large based on this method, creating a systematic overestimate of the true value of $\lambda_n$.
 
\section{Results and fit}
We ran our calculation for critical values of $\lambda$ between $0.5$ and $1000$, using a $\Delta \rho$ of 0.1 and a $\rho_{\infty}$ of $40 \lambda$ and $80 \lambda$. We found 35 critical values of $\lambda$, plotted in figures \ref{fig:loglog} and \ref{fig:quadfit}. The values of $\lambda_n$ along with errors are given in table \ref{tab:l0}. At a quick glance, our data looked pretty quadratic -- just what we expected. In order to explore this further, we attempted to fit the data to functions.

\begin{table}[h]
\centering
	\begin{tabular}{c|c||c|c||c|c}
		$n$ & $\lambda_n$ & $n$ & $\lambda_n$ & $n$ & $\lambda_n$ \\
		\hline
		1	& 0.843	$\pm$ 0.002	& 13	& 133.06$\pm$ 0.05	& 25	& 492.0	$\pm$ 0.4\\
		2	& 3.227	$\pm$ 0.002	& 14	& 154.30$\pm$ 0.06	& 26	& 532.1	$\pm$ 0.5\\
		3	& 7.175	$\pm$ 0.003	& 15	& 177.11$\pm$ 0.08	& 27	& 573.9	$\pm$ 0.5\\
		4	& 12.691$\pm$ 0.003	& 16	& 201.50$\pm$ 0.10	& 28	& 617.2	$\pm$ 0.6\\
		5	& 19.777$\pm$ 0.005	& 17	& 227.5	$\pm$ 0.1	& 29	& 662.2	$\pm$ 0.7\\
		6	& 28.434$\pm$ 0.007	& 18	& 255.0	$\pm$ 0.1	& 30	& 708.7	$\pm$ 0.8\\
		7	& 38.663$\pm$ 0.009	& 19	& 284.1	$\pm$ 0.2	& 31	& 756.8	$\pm$ 0.9\\
		8	& 50.46	$\pm$ 0.01	& 20	& 314.8	$\pm$ 0.2	& 32	& 806	$\pm$ 1\\
		9	& 63.84	$\pm$ 0.02	& 21	& 347.1	$\pm$ 0.2	& 33	& 858	$\pm$ 1\\
		10	& 78.78	$\pm$ 0.02	& 22	& 380.9	$\pm$ 0.3	& 34	& 911	$\pm$ 1\\
		11	& 95.30	$\pm$ 0.03	& 23	& 416.4	$\pm$ 0.3	& 35	& 965	$\pm$ 1\\
		12	& 113.39$\pm$ 0.04	& 24	& 453.4	$\pm$ 0.4	& 		& 
	\end{tabular}
	\caption{Numerically obtained critical values of $\lambda_n$ for $l = 0$, with uncertanties}
	\label{tab:l0}
\end{table}

Because we were looking for a relationship like $\lambda_n = a n^2$, we first tried a weighted linear fit to a log-log plot (figure \ref{fig:loglog}). Our fit gave
\eqn{
log(\lambda_n) = -(0.2246 \pm 0.0003) + (1.9945 \pm 0.0001) \log(n)\mbox{.}
\label{eq:logfit}
} 
The coefficient of the $\log(n)$ term isn't quite 2, and while the fit looks good, it is inaccurate, especially for low values of $n$. When we checked the fit against our measured error bars, the line does not pass through many of them (figure \ref{fig:logerr}). In addition, the residuals seemed to follow a clear pattern, suggesting something systematically wrong with either the data or the fit. From this, we conclude that the dependence is not a pure power law at low $n$.
\begin{figure}[h]
\centering
\includegraphics{Figures/loglog}
\caption[Log-log fit]{Linear fit to a log-log plot of the data.}
\label{fig:loglog}
\end{figure}

\begin{figure}[h]
\centering
\includegraphics{Figures/logerr}
\caption[Residuals of log-log fit]{This plot shows the residuals of the fit (black dots) compared with the error bars for the data points (shaded region). Very few of the black dots lie inside the shaded region, indicating a poor fit.}
\label{fig:logerr}
\end{figure}
Based on this, we decided to try a fit to the equation $\lambda_n = A + B n + C n^2$ (figure \ref{fig:quadfit}). We find a general equation of 
\eqn{
\lambda_n = (0.058 \pm 0.001) + (0.0518 \pm 0.0001) n + ( 0.785389 \pm 0.000003) n^2\mbox{,}
}
which matches the data much better. Residuals for this fit looked much better, with more falling inside the error bars (figure \ref{fig:quaderr}). The residuals roughly follow the lower edge of the error bars, which is consistent with the way we asymptotically approach true values.
\begin{figure}[h]
\centering
\includegraphics{Figures/quadfit}
\caption[Quadratic fit]{Our data again, this time fit to a quadratic.}
\label{fig:quadfit}
\end{figure}

\begin{figure}[h]
\centering
\includegraphics{Figures/quaderr}
\caption[Residuals for quadratic fit]{These are the residuals (black dots) of the quadratic fit compared with the error bars of the data points (shaded region). Many more residuals lay inside the error bars. Those that did not were skewed towards the end of the curve, where the numerical data collected is less accurate.}
\label{fig:quaderr}
\end{figure}

%We computed the average error in $\lambda_n$ based on the fit to check the goodness of fit. The calculated $\Delta \lambda_n$ was $0.0018454$ which is much lower than even our smallest $\Delta \lambda_n$ (0.0275888 for $\lambda_1$). This indicates that the fit is good. The difference between these may suggest that our error estimations are too large, but may also be due to the likelihood that our errors are largely systematic rather than random.

In comparing to published values, we are hindered by the fact that most treatments of the Yukawa potential focus on the energy spectra of states with low $\lambda$ and $n$. However, there do exist published values for $\lambda_1$. Gomes, Chacham, and Mohallem find  $\lambda_1$ to be $0.83990399 \pm 0.00000003$, obtained using a linear combination of atomic orbitals method\cite{PhysRevA.50.228}.

This result sits just at the edge of the error bar for our results. We find $\lambda_1$ to be at least $0.841$, where they find it to be $0.840$ (rounding to the same decimal place). The difference is extremely small (0.01\%), and our estimate is larger, which is consistent with our understanding of the systematic nature of our numerical errors. 

\section{Angular Momentum}

Considering states with nonzero angular momentum is simple. Writing our wave functions as
\eqn{
\psi_{n,l,m} (\rho,\theta,\phi) = \frac{1}{\rho}u_{n,l}(\rho) Y_{l,m}(\theta, \phi)
}
the radial and angular portions separate. We update our radial equation to take into account angular momentum,
\eqn{
-u'' - \left(\frac{2}{\rho}e^{-\rho/\lambda} + \frac{l(l+1)}{\rho^2}\right) u = E u
\label{eq:angmomentum}
}
and again discretize it using \eqref{f2difference}. 

Following the same procedure as before, we can obtain critical values of $\lambda$ for $l = 1$ to $10$. These are plotted in figure \ref{fig:allthelcs} along with the $l = 0$ data from the previous section.

To obtain these curves, we chose to fix $\Delta \rho$ at $0.1$, and set $\rho_{\infty}$ to $80 \lambda$.

\begin{figure}[h]
\centering
\includegraphics{Figures/allthelcs}
\caption[Plots of $\lambda_n$ versus $n$ for $l = 0$ to $10$]{These curves show the relationship between $\lambda_n$ and $n$ for $l = 0$ to 10. In this plot, $n$ is the radial quantum number. The principal quantum number is now $l + n$.}
\label{fig:allthelcs}
\end{figure}

Again, all the curves neatly fit a quadratic, giving fit functions in the form $\lambda_n = A + B n + C n^2$. The constants of the quadratic are dependent on $l$ and listed in table \ref{tab:fitparams}. $B$ and $C$ appear to vary linearly with $l$, while $A$ seems to vary quadratically. This is shown in figure \ref{fig:parameters}.

\begin{figure}[h]
\centering \subfloat[][]{\label{paras:as}
\includegraphics[width=0.3\textwidth]{Figures/poly-as}}
\subfloat[][]{\label{paras:bs}
\includegraphics[width=0.3\textwidth]{Figures/poly-bs}}  
\subfloat[][]{\label{paras:cs}
\includegraphics[width=0.3\textwidth]{Figures/poly-cs}}% \hspace{8pt}%
\caption[Fit parameters for $l = 0$ to 10]{Three graphs showing the changes in fit parameters $a$, $b$, and $c$ as a function of $l$}
\label{fig:parameters}
\end{figure}

\begin{table}[h]
	\centering
	\caption{The fit parameters A, B, and C (with uncertainties) for $l = 0$ to 10.}
	\label{tab:fitparams}
	\begin{tabular}{r|ccc}
		l	& A	 	& B	 	& C	 \\
		\hline
		0	& 0.058	$\pm$ 0.001	& 0.0518	$\pm$ 0.0001	& 0.785389	$\pm$ 0.000003\\
		1	& 2.08	$\pm$ 0.03	& 1.872	$\pm$ 0.004	& 0.7834	$\pm$ 0.0001\\
		2	& 6.70	$\pm$ 0.06	& 3.762	$\pm$ 0.008	& 0.7810	$\pm$ 0.0002\\
		3	& 13.97	$\pm$ 0.07	& 5.67	$\pm$ 0.01	& 0.7787	$\pm$ 0.0003\\
		4	& 23.92	$\pm$ 0.08	& 7.59	$\pm$ 0.01	& 0.7763	$\pm$ 0.0004\\
		5	& 36.55	$\pm$ 0.09	& 9.52	$\pm$ 0.01	& 0.7741	$\pm$ 0.0004\\
		6	& 51.84	$\pm$ 0.09	& 11.46	$\pm$ 0.01	& 0.7714	$\pm$ 0.0005\\
		7	& 69.88	$\pm$ 0.09	& 13.40	$\pm$ 0.01	& 0.7693	$\pm$ 0.0005\\
		8	& 90.62	$\pm$ 0.08	& 15.34	$\pm$ 0.01	& 0.7672	$\pm$ 0.0006\\
		9	& 114.08	$\pm$ 0.08	& 17.28	$\pm$ 0.01	& 0.7652	$\pm$ 0.0006\\
		10	& 140.22	$\pm$ 0.07	& 19.23	$\pm$ 0.01	& 0.7626	$\pm$ 0.0006
	\end{tabular}
\end{table}

We note that when $l$ is non-zero, the quantum number $n$ is actually the \emph{radial} quantum number. Our numerical procedure finds the lowest energy state for a given value of $l$, and labels that state with $n = 1$. The next is then $n = 2$, and so on. The principal quantum number then becomes $n_p = n + l$. Thus, at principal quantum number $n_p = 1$, we require that $l = 0$. At $n_p = 2$, we can have $l = 0$ or $l = 1$. Accordingly, the bound states at higher $l$ start further away from the origin at a higher value of $\lambda$. 
This explains some of the dependence of $A$, $B$, and $C$ on $l$ if we assume that $\lambda$ should actually be quadratic in $n_p$. In this case we can write
\eqn{
\lambda_{n_p} = \tilde{A} + \tilde{B} n_p + \tilde{C} n_p^2 = \tilde{A} + \tilde{B} (n + l) + \tilde{C} (n + l)^2\mbox{.}
}
When we do this, we obtain the plot of $\lambda_{n_p}$s shown in figure \ref{fig:alllcs-principal}.

\begin{figure}[h]
\centering
\includegraphics{Figures/alllcs-principal}
\caption[Plots of $\lambda_n$ versus $n + l$ for $l = 0$ to $10$]{These curves show the relationship between $\lambda_n$ and $n_p = n + l$ for $n_p = 0$ to 35.}
\label{fig:alllcs-principal}
\end{figure}

These parameters are closer to constant, but we can see that they still have some remaining dependence on $l$.

%\clearpage %% starts a new page and stops trying to place floats such as tables and figures

%
%
\chapter*{Conclusion}
         \addcontentsline{toc}{chapter}{Conclusion}
	\chaptermark{Conclusion}
	\markboth{Conclusion}{Conclusion}
	\setcounter{chapter}{4}
	\setcounter{section}{0}

The Yukawa potential has only a finite number of bound states. The potential decreases to zero quickly outside of a region described by the length scale $\lambda$, a ratio between the exponential cutoff length scale $\ell$ and the ``Bohr radius'' $\alpha$. Any state sitting too far from the origin compared to this length scale cannot be bound, because there is not enough force to bind it. This implies that the number of bound states increases with increasing $\lambda$, and we can define a critical value $\lambda_n$ as the value of $\lambda$ at which the $n$th bound state appears.
We first used the Bohr model to get an approximate relationship between the number of bound states of the Yukawa potential and the value of the length scale. We then obtained the same relationship directly from the potential using the finite difference method. As with other central force potentials, the Bohr method gave a reasonable approximation, but the numerical picture was more subtle. 

We found that the relationship between the number of bound states $n$ and the non-dimensionalized length scale $\lambda$ was of the form $\lambda_n = A(l) + B(l)n + C(l)n^2$, where $A$, $B$, and $C$ are constants dependent on the quantum number $l$. 

While this method was intended to look at the number of bound states, not the energies, a possible extension of this work would be to obtain an energy spectrum at some value of $\lambda$. However, in order to do this, one would need a different routine for finding the eigenvalues of the Hamiltonian matrix, as the continuous eigenvalues above zero create computational challenges.

We could also extend this work to investigate the unstable bound states -- those states with $E > 0$ that are still partially bound. These states should primarily occur at higher values of $\lambda$, and should be distinguishable from pure scattering states. 


%
%If you feel it necessary to include an appendix, it goes here.
\appendix
\chapter{Numerical Methods}

\section{Rootfinder}

To actually locate the critical values of lambda, we need to find the place where a new bound state appears. Numerically, this looks like a new negative eigenvalue in our list, which suggests we approach the problem with rootfinding. Our function has a ``zero'' whenever the number of negative eigenvalues increases.

Here it is in \emph{Mathematica} code:
\begin{Verbatim}[commandchars=\\\{\}, codes={\catcode`$=3}]
Critical[$\lambda$_, consts_] :=  Module[\{H, res, d$\rho$, $\rho$inf, nn\},
    d$\rho$ = consts[[1]];
    $\rho$inf = Round[consts[[2]]];
    nn = Round[$\rho$inf/d$\rho$];
    H = SparseArray[\{\{n_, n_\} -> 
        2./d$\rho$^2 - 2./(n*d$\rho$) $e$^(-n*d$\rho$/$\lambda$) + (
        l (l + 1))/(n*d$\rho$)^2, \{n_, m_\} /; Abs[n - m] == 1 -> -1./
        d$\rho$^2\}, \{nn, nn\}, 0.];
    res = Sort[Eigenvalues[H, -25]];
    Return[res]
  ]
\end{Verbatim}
This function takes a value of lambda and returns the some number of eigenvalues (here, 25)  of the hamiltonian which have the lowest absolute value. $nn$ is the matrix size, d$\rho$ is the step size, and $\rho$inf is numerical infinity for the problem.

The results of this calculation are extremely sensitive to $\rho$inf, and a larger $\rho$inf makes for less error. However, for larger values of $\rho$inf, more positive eigenvalues are found near 0, because the positive eigenvalues are in a continuous spectrum. As the grid becomes a better approximation of reality, more of these are picked up. This squeezes the negative eigenvalues (whose spectrum starts near -1) out, especially at higher values of $\lambda$. In order to ensure that there are *some* negative eigenvalues, we therefore pin $\rho$inf to some constant factor times $\lambda$. 

Increasing the number of negative eigenvalues found will also ensure we don't miss negative eigenvalues, but is very expensive. When searching for initial eigenvalues, we chose to get the 100 with the lowest absolute value in order to increase accuracy, but for the main calculation, 25 was fine.

The rootfinder actually used was a recursive bisection routine. In this rootfinder, we stopped recursing when the difference between the two endpoints was less than some $\delta$, instead of when the values of the function at the endpoints were within $\epsilon$. Because our function just looks at the number of negative eigenvalues, $f(\lambda_f) - f(\lambda_0)$ should always be equal to $1$ in a range with a zero and $\epsilon$ is meaningless. 

Again, in \emph{Mathematica} code:

\begin{Verbatim}[commandchars=\\\{\}, codes={\catcode`$=3}]
Bisect[F_, consts_, x0_, xf_, $\epsilon$_] := Module[\{mid, val, lnegs, vam, mnegs, vah\},
    mid = x0 + Abs[x0 - xf]/2;
    Print["Bisecting. Midpoint is " <> ToString[mid]]; (*progress indicator*)
    val = F[x0, consts];
    lnegs = CountNegs[val];
    vam = F[mid, consts];
    mnegs = CountNegs[vam];
    If[xf - x0 > $\epsilon$,
        If[lnegs < mnegs || (lnegs == mnegs && val[[1]] < vam[[1]]), 
            Return[Bisect[F, consts, x0, mid, $\epsilon$]];, 
            Return[Bisect[F, consts, mid, xf, $\epsilon$]]; 
     ];,
   vah = F[xf, consts];
   Return[{mid, lnegs, CountNegs[vah]}];
   ]
]
\end{Verbatim}

Again, when we bisect, we are looking at the number of negative eigenvalues as a function of lambda. In this routine, we assume multiple zeroes, and keep bisecting any range with a zero in it.

The complicated conditional for bisection is meant to deal with the problem of negative eigenvalues getting squeezed out by positive ones. Given a half of the input range, it bisects that half if either the number of negative eigenvalues increases (a regular zero), or the number of negative eigenvalues stays constant, but the value of the first negative eigenvalue jumps. This indicates that one negative eigenvalue was squeezed out, but another was added. In general, as $\lambda$ increases, the negative eigenvalues should decrease in value (coming closer and closer to the Coulomb energies). 

To find the ranges to bisect, we modify the rootfinder so it keeps bisecting as long as there is a zero in one of the ranges, and returns a list of points instead of just one. For efficiency's sake, this calculation is run at a low $\rho_{\infty}$, and a range is returned instead of a single point. That range can then be bisected more thoroughly in order to identify the most likely point. 

As is clear from the code, the function doing the actual work is just Mathematica's Eigenvalues[ ]. Because our Hamiltonian matrix only has entries on the diagonal and off-diagonals, we can define it as a SparseArray to speed up calculation. 

Finding a single $\lambda_{n}$ takes quite a bit of time -- it's expensive to find the eigenvalues of a large matrix, even such a sparse one as ours. We would like to study a wide range of $\lambda$s, and the computation begins to look prohibitively expensive.

However, the problem is embarrassingly parallel. Each bisection to find a value of $\lambda_n$ is independent of all the others, allowing us to easily split this problem into chunks. We used Mathematica's Lightweight Grid to run the computation on a cluster of machines. Distributing these calculations among multiple cores speeds the process up by about 50\% per core added. 

\section{Fits}
\subsection{Linear}
Our weighted linear least-squares fit was computed following the methods outlined in Taylor\cite{taylor_introduction_1997}. We first compute weights for each point, based on its error, then use that to weight the least-squares calculation.

Here it is in Mathematica code:
\begin{Verbatim}
weights = Table[1/(logerrs[[i]]^2), {i, 1, Length[logerrs]}];

x[i_] := loglog[[i, 1]];
y[i_] := loglog[[i, 2]];
w[i_] := weights[[i]]; 
len = Length[loglog]

Delta = Sum[w[[i]], {i, len}]*Sum[w[[i]]*x[i]^2, {i, len}] -
	(Sum[w[[i]]*x[i], {i, len}])^2;

A = 1/Delta*(
	Sum[w[i]*x[i]^2, {i, len}] *Sum[w[i]*y[i], {i, len}] - 
	Sum[w[i]*x[i], {i, len}]*Sum[w[i]*x[i]*y[i], {i, len}])
	
B = 1/Delta*(
	Sum[w[i], {i, len}]*Sum[w[i]*x[i]*y[i], {i, len}] - 
	Sum[w[i]*x[i], {i, len}]*Sum[w[i]*y[i], {i, len}])
\end{Verbatim}

\subsection{Quadratic}
For our quadratic fit, we decided to ignore data weights as the errors were largely on similar orders of magnitude.  Instead of calculating the coefficients of the fit independently, we did it with a matrix inversion, because the expressions for each component are complicated. More precisely, we solved the equation
\eqn{
\left(
\begin{array}{ccc}
N & \sum n_i &  \sum n_i^2 \\
\sum n_i & \sum n_i ^2 & \sum n_i^3 \\
\sum n_i ^2 & \sum n_i^3 & \sum n_i ^4
\end{array}
\right)
\left(
\begin{array}{c}
A \\
B \\
C \\
\end{array}
\right)
=
\left(
\begin{array}{c}
\sum \lambda_i \\
\sum n_i \lambda_i \\
\sum n_i^2 \lambda_i
\end{array}
\right)
}
for $A$, $B$, and $C$. For the matrix inversion, we just used the one built into Mathematica. 

When we turned this into a weighted fit, we picked up some extra $\Delta \lambda_i$ terms:
\eqn{
\left(
\begin{array}{ccc}
\sum \frac{1}{\Delta \lambda_i^2} & \sum \frac{n_i}{\Delta \lambda_i^2} &  \sum \frac{n_i^2}{\Delta \lambda_i^2} \\
\sum \frac{n_i}{\Delta \lambda_i^2} & \sum \frac{n_i ^2}{\Delta \lambda_i^2} & \sum \frac{n_i^3}{\Delta \lambda_i^2} \\
\sum \frac{n_i ^2}{\Delta \lambda_i^2} & \sum \frac{n_i^3}{\Delta \lambda_i^2} & \sum \frac{n_i ^4}{\Delta \lambda_i^2}
\end{array}
\right)
\left(
\begin{array}{c}
A \\
B \\
C \\
\end{array}
\right)
=
\left(
\begin{array}{c}
\sum \frac{\lambda_i}{\Delta \lambda_i^2} \\
\sum \frac{n_i \lambda_i}{\Delta \lambda_i^2} \\
\sum \frac{n_i^2 \lambda_i}{\Delta \lambda_i^2}
\end{array}
\right)
}


\chapter{Data}
\begin{table}[h]
\begin{adjustwidth}{-0.5in}{-0.5in}
	\caption{All the critical values $\lambda_n$ for non-zero $l$.}
	\label{tab:alldata}
	\begin{tabular}{c|cccccccccc}
		$n$	& $l = 1$	& $l = 2$	& $l = 3$	& $l = 4$	& $l = 5$	& $l = 6$	& $l = 7$	& $l = 8$	& $l = 9$	& $l = 10$\\
		\hline
		1	& 4.541	& 10.947	& 20.068	& 31.904	& 46.459	& 63.730	& 83.720	& 106.428	& 131.854	& 159.999\\
		2	& 8.872	& 17.210	& 28.257	& 42.018	& 58.496	& 77.691	& 99.604	& 124.235	& 151.583	& 181.650\\
		3	& 14.730	& 24.984	& 37.949	& 53.630	& 72.028	& 93.143	& 116.977	& 143.528	& 172.798	& 204.786\\
		4	& 22.130	& 34.285	& 49.159	& 66.752	& 87.064	& 110.096	& 135.847	& 164.318	& 195.507	& 229.414\\
		5	& 31.080	& 45.121	& 61.894	& 81.392	& 103.614	& 128.558	& 156.224	& 186.609	& 219.715	& 255.540\\
		6	& 41.585	& 57.501	& 76.163	& 97.559	& 121.684	& 148.536	& 178.112	& 210.410	& 245.430	& 283.170\\
		7	& 53.647	& 71.427	& 91.970	& 115.257	& 141.281	& 170.035	& 201.517	& 235.725	& 272.655	& 312.308\\
		8	& 67.270	& 86.905	& 109.321	& 134.493	& 162.409	& 193.061	& 226.445	& 262.558	& 301.396	& 342.959\\
		9	& 82.455	& 103.937	& 128.219	& 155.268	& 185.071	& 217.617	& 252.899	& 290.913	& 331.657	& 375.127\\
		10	& 99.204	& 122.525	& 148.666	& 177.588	& 209.272	& 243.706	& 280.883	& 320.795	& 363.440	& 408.814\\
		11	& 117.516	& 142.670	& 170.664	& 201.453	& 235.014	& 271.333	& 310.399	& 352.206	& 396.750	& 444.025\\
		12	& 137.394	& 164.375	& 194.217	& 226.867	& 262.300	& 300.499	& 341.451	& 385.150	& 431.588	& 480.762\\
		13	& 158.838	& 187.641	& 219.325	& 253.832	& 291.132	& 331.207	& 374.042	& 419.628	& 467.959	& 519.028\\
		14	& 181.848	& 212.468	& 245.990	& 282.349	& 321.512	& 363.459	& 408.172	& 455.643	& 505.863	& 558.826\\
		15	& 206.425	& 238.857	& 274.212	& 312.419	& 353.442	& 397.256	& 443.846	& 493.198	& 545.303	& 600.156\\
		16	& 232.569	& 266.809	& 303.994	& 344.045	& 386.923	& 432.602	& 481.063	& 532.293	& 586.282	& 643.022\\
		17	& 260.281	& 296.325	& 335.336	& 377.226	& 421.956	& 469.496	& 519.826	& 572.931	& 628.800	& 687.425\\
		18	& 289.560	& 327.406	& 368.238	& 411.965	& 458.543	& 507.941	& 560.136	& 615.113	& 672.860	& 733.367\\
		19	& 320.407	& 360.051	& 402.701	& 448.262	& 496.685	& 547.937	& 601.995	& 658.841	& 718.462	& 780.849\\
		20	& 352.823	& 394.261	& 438.727	& 486.118	& 536.382	& 589.486	& 645.403	& 704.116	& 765.609	& 829.873\\
		21	& 386.807	& 430.037	& 476.315	& 525.533	& 577.636	& 632.588	& 690.363	& 750.939	& 814.302	& 880.440\\
		22	& 422.359	& 467.379	& 515.467	& 566.508	& 620.447	& 677.245	& 736.874	& 799.311	& 864.541	& 932.551\\
		23	& 459.480	& 506.287	& 556.182	& 609.045	& 664.817	& 723.458	& 784.938	& 849.233	& 916.328	& -\\
		24	& 498.170	& 546.761	& 598.461	& 653.142	& 710.745	& 771.227	& 834.555	& 900.707	& 969.664	& -\\
		25	& 538.429	& 588.802	& 642.304	& 698.802	& 758.233	& 820.552	& 885.727	& 953.733	& -	& -\\
		26	& 580.257	& 632.410	& 687.712	& 746.023	& 807.280	& 871.435	& 938.455	& -	& -	& -\\
		27	& 623.654	& 677.585	& 734.684	& 794.808	& 857.888	& 923.877	& -	& -	& -	& -\\
		28	& 668.620	& 724.328	& 783.223	& 845.156	& 910.057	& -	& -	& -	& -	& -\\
		29	& 715.156	& 772.637	& 833.326	& 897.067	& 963.787	& -	& -	& -	& -	& -\\
		30	& 763.261	& 822.515	& 884.995	& 950.542	& -	& -	& -	& -	& -	& -\\
		31	& 812.935	& 873.960	& 938.231	& -	& -	& -	& -	& -	& -	& -\\
		32	& 864.179	& 926.973	& -	& -	& -	& -	& -	& -	& -	& -\\
		33	& 916.992	& -	& -	& -	& -	& -	& -	& -	& -	& -\\
		34	& 971.375	& -	& -	& -	& -	& -	& -	& -	& -	& -\\
		35	& -	& -	& -	& -	& -	& -	& -	& -	& -	& -
	
	\end{tabular}
	\end{adjustwidth}
\end{table}


%      \chapter{The Second Appendix, for Fun}
%
%
%%This is where endnotes are supposed to go, if you have them.
%%I have no idea how endnotes work with LaTeX.
%
\backmatter % backmatter makes the index and bibliography appear properly in the t.o.c...
%
% Make my bibliography be called "Bibliography" and not "References" (or "Works Cited" or...):
%% \renewcommand{\bibname}{Works Cited}

 \bibliographystyle{revtex}
%FIXME: should be apsrev.
  
% there are a variety of styles available; 
%% replace ``plainnat'' with the style of choice. You can refer to files in the bsts or APA 
%% subfolder, e.g. 
%% \bibliographystyle{APA/apa-good}  % or
%% \bibliographystyle{bsts/mla-good} 
%
%% if you're using bibtex, the next line forces every entry in the bibtex file to be included
%% in your bibliography, regardless of whether or not you've cited it in the thesis.
   \nocite{*}
   \bibliography{em_thesis}

% Finally, an index would go here... but it is also optional.
\end{document}
