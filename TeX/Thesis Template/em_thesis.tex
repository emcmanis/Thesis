% This is the Reed College LaTeX thesis template. Most of the work 
% for the document class was done by Sam Noble (SN), as well as this
% template. Later comments etc. by Ben Salzberg (BTS). Additional
% restructuring and APA support by Jess Youngberg (JY).
% Your comments and suggestions are more than welcome; please email
% them to cus@reed.edu
%
% See http://web.reed.edu/cis/help/latex.html for help. There are a 
% great bunch of help pages there, with notes on
% getting started, bibtex, etc. Go there and read it if you're not
% already familiar with LaTeX.
%
% Any line that starts with a percent symbol is a comment. 
% They won't show up in the document, and are useful for notes 
% to yourself and explaining commands. 
% Commenting also removes a line from the document; 
% very handy for troubleshooting problems. -BTS

% As far as I know, this follows the requirements laid out in 
% the 2002-2003 Senior Handbook. Ask a librarian to check the 
% document before binding. -SN

%%
%% Preamble
%%
% \documentclass{<something>} must begin each LaTeX document
\documentclass[12pt,twoside]{reedthesis}
% Packages are extensions to the basic LaTeX functions. Whatever you
% want to typeset, there is probably a package out there for it.
% Chemistry (chemtex), screenplays, you name it.
% Check out CTAN to see: http://www.ctan.org/
%%
\usepackage{graphicx,latexsym} 
\usepackage{amssymb,amsthm,amsmath}
\usepackage{longtable,booktabs,setspace} 
%\usepackage{chemarr} %% Useful for one reaction arrow, useless if you're not a chem major
\usepackage{url}
\usepackage{natbib}
% \usepackage{times} % other fonts are available like times, bookman, charter, palatino

\newcommand{\eqn}[1]{\begin{equation}#1\end{equation}}
\newcommand{\eq}[1]{\begin{align}#1\end{align}}


\title{Quantum Mechanical Bound States of the Yukawa Potential (or some better title)}
\author{Ellen M. McManis}
% The month and year that you submit your FINAL draft TO THE LIBRARY (May or December)
\date{May 2012}
\division{Mathematics and Natural Sciences}
\advisor{Nelia Mann}
%If you have two advisors for some reason, you can use the following
%\altadvisor{Your Other Advisor}
%%% Remember to use the correct department!
\department{Physics}
% if you're writing a thesis in an interdisciplinary major,
% uncomment the line below and change the text as appropriate.
% check the Senior Handbook if unsure.
%\thedivisionof{The Established Interdisciplinary Committee for}
% if you want the approval page to say "Approved for the Committee",
% uncomment the next line
%\approvedforthe{Committee}

\setlength{\parskip}{0pt}
%%
%% End Preamble
%%
%% The fun begins:
\begin{document}

  \maketitle
  \frontmatter % this stuff will be roman-numbered
  \pagestyle{empty} % this removes page numbers from the frontmatter

% Acknowledgements (Acceptable American spelling) are optional
% So are Acknowledgments (proper English spelling)
    \chapter*{Acknowledgements}
	People. Things. Shep, the dog who is currently keeping me company.

% The preface is optional
% To remove it, comment it out or delete it.
%    \chapter*{Preface}
%	This is an example of a thesis setup to use the reed thesis document class.

    \tableofcontents
% if you want a list of tables, optional
    \listoftables
% if you want a list of figures, also optional
    \listoffigures

% The abstract is not required if you're writing a creative thesis (but aren't they all?)
% If your abstract is longer than a page, there may be a formatting issue.
    \chapter*{Abstract}
	Math and computers and stuff gave me results!
		
%	\chapter*{Dedication}
%	You can have a dedication here if you wish.

  \mainmatter % here the regular arabic numbering starts
  \pagestyle{fancyplain} % turns page numbering back on

%The \introduction command is provided as a convenience.
%if you want special chapter formatting, you'll probably want to avoid using it altogether

    \chapter*{Introduction}
         \addcontentsline{toc}{chapter}{Introduction}
	\chaptermark{Introduction}
	\markboth{Introduction}{Introduction}
	% The three lines above are to make sure that the headers are right, that the intro gets included in the table of contents, and that it doesn't get numbered 1 so that chapter one is 1.

% Double spacing: if you want to double space, or one and a half 
% space, uncomment one of the following lines. You can go back to 
% single spacing with the \singlespacing command.
% \onehalfspacing
% \doublespacing

\section{The hydrogen atom}

Before we talk about the Yukawa potential, let's look at a simpler system: the Coulomb potential in the hydrogen atom. Here, a single negatively charged electron is bound to the positively charged nucleus, consisting of a single proton. The time-independent wave equation for that electron is
\eqn{
-\frac{\hbar^2}{2\mu}\nabla^2\psi(r,\phi,\theta) - \frac{e^2}{4\pi \epsilon_0 r}\psi (r,\phi,\theta) = E \psi(r,\phi,\theta)\mbox{.}
\label{eq:TIDSWE-hydrogen}
}
This equation can be solved analytically by separation of variables.\footnote{How much of this should I write out? After writing below this, probably more than this.} While the final wave function will be dependent on all three variables, the energy is just a function of $n$ (the radial quantum number), and is equal to
\eqn{
E_n = -\frac{\mu e^4}{(4 \pi \epsilon_0)^2 2 \hbar ^2 n^2}\mbox{.}
\label{eq:hydrogen-E}
}
This can be rewritten in terms of 
\eqn{
a_0 \equiv \frac{4 \pi \epsilon_0 \hbar ^2}{\mu e^2}\mbox{,}
}
the ``Bohr radius". A constant with units of length, $a_0$ is defined in terms of the other constants in the problem. It is also equal to the radius of the smallest orbit in a Bohr hydrogen atom (the ground state). As such, it sets a natural length scale. Substituting $a_0$ in, the expression for $E_n$ becomes much simpler:
\eqn{
E_n = -\frac{\hbar^2}{a_0^2 \mu n^2}\mbox{.}
\label{eq:hydrogen-Ebohr}
}
These energies are the energies of the bound states of the electron in the hydrogen atom. As $n$ increases, $E_n$ gets closer and closer to zero, but never quite reaches it -- there are an infinite number of bound states possible.
\begin{figure}
\includegraphics[scale=0.75]{hydrogenspectrum.png}
\caption{The energies of the first 10 bound states of the electron in the hydrogen atom}
\label{fig:hspec}
\end{figure}

The ground state wave function is spherically symmetric, meaning that $\psi$ there is only a function of $r$. This function is plotted in Figure \ref{fig:hground}.

\begin{figure}
\includegraphics[scale=0.75]{hydground.png}
\caption{The ground state ($n=1$, $l =0$, $m= 0$) wave function for hydrogen}
\label{fig:hground}
\end{figure}

\section{Comparison with the Yukawa potential}
The Yukawa potential governs the force between protons and neutrons in the nucleus of the atom. It is
\eqn{
V(r) = -\frac{C}{r}e^{-r/l}\mbox{.}
}
When considering this potential, the analogous system to hydrogen is the deuterium atom. One proton and one neutron means you can view the system as a reduced mass orbiting the center of mass. The exponential term provides an effective cutoff once $r$ gets much larger than $l$, as the exponential term drops off quite rapidly. This means the Yukawa potential's range is limited in a way the Coulomb potential's isn't. We expect that this will limit the number of bound states as well, to some number whose $r$s are less than $l$.

The force comes from the exchange of virtual pions between the two nucleons. The limited range is due to the fact that pions have mass; the Coulomb force is mediated by massless virtual photons. The mass of the pion then generates the length scale:
\eqn{
l = \frac{\hbar}{m_{\pi}c} \approx 1.41 \times 10^{-15}\mbox{ m}
}
The strength of the force $C$ is only known experimentally; it has been found to be about $2.96 \times 10^{-6}$ eV*m\cite{ER}\footnote{Get a better citation for this than Eisberg and Resnick}. These values give us an idea of how the force will work in an actual atom, and provide something we can plug in to test our model.
%\clearpage %% starts a new page and stops trying to place floats such as tables and figures
%
%
%\chapter*{Conclusion}
%         \addcontentsline{toc}{chapter}{Conclusion}
%	\chaptermark{Conclusion}
%	\markboth{Conclusion}{Conclusion}
%	\setcounter{chapter}{4}
%	\setcounter{section}{0}
%	
%
%%If you feel it necessary to include an appendix, it goes here.
%    \appendix
%      \chapter{The First Appendix}
%      \chapter{The Second Appendix, for Fun}
%
%
%%This is where endnotes are supposed to go, if you have them.
%%I have no idea how endnotes work with LaTeX.
%
\backmatter % backmatter makes the index and bibliography appear properly in the t.o.c...
%
% Make my bibliography be called "Bibliography" and not "References" (or "Works Cited" or...):
%% \renewcommand{\bibname}{Works Cited}
  \bibliographystyle{plain} % there are a variety of styles available; 
%% replace ``plainnat'' with the style of choice. You can refer to files in the bsts or APA 
%% subfolder, e.g. 
%% \bibliographystyle{APA/apa-good}  % or
%% \bibliographystyle{bsts/mla-good} 
%
%% if you're using bibtex, the next line forces every entry in the bibtex file to be included
%% in your bibliography, regardless of whether or not you've cited it in the thesis.
   \nocite{*}
   \bibliography{em_thesis}

% Finally, an index would go here... but it is also optional.
\end{document}
